\documentclass{beamer}
\usetheme{Copenhagen}
%Information to be included in the title page:
\title{Solution Slides FSR Coding Cup 2023}
\author{Brutenis Gliwa, Nick Kotsakidis, Marian Zuska}
\institute{Universität Rostock}
\date{2023}

\begin{document}

\frame{\titlepage}

\begin{frame}
    \frametitle{Pizza}
    \begin{itemize}
        \item Sponsored by PLANET AI GmbH (planet-ai.de):
        \item 3*Margharita
        \item 2*Salami
        \item 2*Hawaii
        \item 1*Beef
        \item 1*Vegetarisch
        \item 1*Mozarella
        \item 1*Hot Dog
    \end{itemize}
\end{frame}

\begin{frame}
    \frametitle{Intuitive Citations}
    \begin{block}{Problem}
        \begin{itemize}
            \item Given a list of names. Print the lexicographically smallest surname and add \texttt{" et al."}.
        \end{itemize}
    \end{block} \pause
    \begin{block}{Solution} 
        \begin{itemize}
            \item Remove part before space (forename). \pause
            \item Find the lexicographically smallest string by sorting and taking the first element. \pause
            \item Print string + \texttt{" et al."}.
        \end{itemize}
    \end{block} 
\end{frame}

\begin{frame}
    \frametitle{Compilers Brackets}
    \begin{block}{Problem}
        \begin{itemize}
            \item Check if given bracket pattern is valid.
            \item {\tt"\{\{\}\{\}"} is not valid.
            \item {\tt"\{\{\}\{\{\}\}\}"} is valid.
        \end{itemize}
    \end{block} \pause
    \begin{block}{Solution}
        \begin{itemize}
            \item Count number of currently open brackets.
            \pause
            \item Begin with {\tt open = 0}
            \pause
            \item {\tt"\{"} $\rightarrow$ {\tt open++}
            \pause
            \item {\tt"\}"} $\rightarrow$ {\tt open--}
            \pause
            \item Pattern invalid if {\tt open < 0} at any time.
            \pause
            \item Pattern invalid if {\tt open != 0} in the end.
        \end{itemize}
    \end{block} 
\end{frame}

\begin{frame}
    \frametitle{Fascinating Books}
    \begin{block}{Problem}
        \begin{itemize}
            \item Check if given list of strings contains every letter of the alphabet.
        \end{itemize}
    \end{block} \pause
    \begin{block}{Solution}
        \begin{itemize}
            \item Can concatenate strings and solve for a single string. \pause
            \item For each character:
            \begin{itemize}
            \item If character is letter: Add as lowercase to set. \pause
            \end{itemize}
        \item Check if length of set is 26.
        \end{itemize}
    \end{block} \pause
    \begin{block}{Gotchas}
        \begin{itemize}
            \item Capitalization does not matter. \pause
            \item Strings do not only contain letters.
        \end{itemize}
    \end{block} 
\end{frame}

\begin{frame}
    \frametitle{Leaderboard Prediction}
    \begin{block}{Problem}
        \begin{itemize}
            \item Given the times you need to solve each of the $n$ problems.
            \item Determine the minimal penalty you can get on the contest.
        \end{itemize}
    \end{block} \pause
    \begin{block}{Solution}
        \begin{itemize}
            \item The time you needed for the first problem will be added to the penalty of all problems you solve.
            \pause
            \item It is always best to solve shortest problems first.
            \pause
            \item Greedy solution: Sort problems by length, then simulate.
        \end{itemize}
    \end{block} 
\end{frame}

\begin{frame}
    \frametitle{Bicycle Lock}
    \begin{block}{Problem}
        \begin{itemize}
            \item Input: Initial lock position $I$ and final lock position $F$ of length $n$.
            \item Move to final position by always turning two consecutive dials at once.
        \end{itemize}
    \end{block} \pause
    \begin{block}{Solution}
        \begin{itemize}
            \item Dial 1 can only be turned by turning dials 1 and 2.
            \pause
            \item It needs to be turned from $I_1$ to $I_2$.
            \pause
            \item After turning that, we have a new subproblem of length $n-1$.
            \pause
            \item We can solve this recursively.
        \end{itemize}
    \end{block} \pause
    \begin{block}{Gotchas}
        \begin{itemize}
            \item Always two ways to turn dials: Clockwise or Anti-clockwise.
            \pause
            \item Need to check if last dial is at the right positition in the end.
        \end{itemize}
    \end{block} 
\end{frame}

\begin{frame}
    \frametitle{Aquarium  Maze}
    \begin{block}{Problem}
        \begin{itemize}
            \item Input: grid of $"."$ and $"\#"$ squares.
            \item Grid is filled with water from the top.
            \item Water can move down, left and right.
        \end{itemize}
    \end{block} \pause
    \begin{block}{Solution}
        \begin{itemize}
            \item Can simulate water by starting at a top square and then traversing the graph e.g. using BFS or DFS.
            \pause
            \item For each top square:
            \begin{itemize}
                \item if square $== "."$ and not yet visited:
                \begin{itemize}
                    \item add 1 to answer.
                    \pause
                    \item visit left, right and bottom neighbour recursively.
                \end{itemize}
            \end{itemize}
        \end{itemize}
    \end{block} \pause
    \begin{block}{Gotchas}
        \begin{itemize}
            \item Need to start once at every point on the top.
            \pause
            \item Otherwise we might miss some air bubbles.
        \end{itemize}
    \end{block} 
\end{frame}

\begin{frame}
    \frametitle{Hidden Words}
    \begin{block}{Problem}
        \begin{itemize}
            \item Given $n$ strings $s_1$ ... $s_n$.
            \item Find a string $S$ that contains all $n$ strings in consecutive order and where no character is part of 3 strings.
        \end{itemize}
    \end{block} \pause
    \begin{block}{Solution}
        \begin{itemize}
            \item Start with two first strings $s_1$ and $s_2$.
            \pause
            \item If $s_2$ starts with $s_1$:
            \pause
            \item Add $s_1$ to $S$, continue with rest of $s_2$ and next string.
            \pause
            \item Else: remove first letter of $s_1$ and repeat process.
        \end{itemize}
    \end{block} 
\end{frame}

\begin{frame}
    \frametitle{Jolly Fishing}
    \begin{block}{Problem}
        \begin{itemize}
            \item Choose if fishers are allowed to fish on each day of the year.
            \item Fishes will only reproduce when not being fished.
            \item After the year, there need to be at least as many fishes as in the beginning.
        \end{itemize}
    \end{block} \pause
    \begin{block}{Solution}
        \begin{itemize}
            \item It is always better to not allow fishing for the first part of the year and then allow fishing for the rest of the year.
            \pause
            \item (If we did not allow fishing after a day where we did, we would get a higher score by swapping the two days).
            \pause
            \item Thus we only need to determine the day we start allowing fishing.
            \pause
            \item Can simulate every 365 possible days (or use Ternary Search).
        \end{itemize}
    \end{block} 
\end{frame}

\begin{frame}
    \frametitle{Dam Construction}
    \begin{block}{Problem}
        \begin{itemize}
            \item Given $(n_1, n_2, n_4)$ lego bricks of size 1, 2 and 4.
            \item Build the highest wall of width $w$.
        \end{itemize}
    \end{block} \pause
    \begin{block}{Solution}
        \begin{itemize}
            \item Should always use bricks of higher size first to maintain flexibility.
            \pause
            \item Greedy solution by using $n_4$ bricks, then filling up with $n_2$, then with $n_1$.
        \end{itemize}
    \end{block} \pause
    \begin{block}{Gotchas}
        \begin{itemize}
            \item In slower languages (like Python) you need to calculate in 1 step how many bricks of each type you need.
            \item Also the tower can get to heights of $10^9$ which means calculating each row in a single step might still be too slow.
        \end{itemize}
    \end{block} 
\end{frame}

\begin{frame}
    \frametitle{Extravagant Voyage}
    \begin{block}{Problem}
        \begin{itemize}
            \item Given $n$ items with happiness $H$ and volume $V$.
            \item Choose items with cumulative weight $w$.
            \item Also called 0-1-Knapsack.
        \end{itemize}
    \end{block} \pause
    \begin{block}{Solution}
        \begin{itemize}
            \item Recursive DP solution:
            \item Go through $n$ items and start with remaining weight $r$ = $w$.
            \pause
            \item Recursively solve:
            \pause
            \begin{itemize}
                \item taking item: $r$ -= $W_i$; $h$ += $H_i$.
                \pause
                \item leaving item: $r$, $h$ unchanged.
            \end{itemize}
            \pause
            \item Save states in dp-table.
        \end{itemize}
    \end{block} \pause
    \begin{block}{Gotchas}
        \begin{itemize}
            \item Not using a dp-table results in time limit exceeded.
        \end{itemize}
    \end{block} 
\end{frame}

\begin{frame}
    \frametitle{Going Home}
    \begin{block}{Problem}
        \begin{itemize}
            \item Given different train connections, find the shortest time to get from Rostock to the given city.
        \end{itemize}
    \end{block} \pause
    \begin{block}{Solution}
        \begin{itemize}
            \item Can be solved using any shortest path algorithms that allows for edge weights e.g. Dijkstra.
        \end{itemize}
    \end{block} \pause
    \begin{block}{Gotchas}
        \begin{itemize}
            \item The input is rather complicated and has to be parsed into a graph structure first.
            \pause
            \item Need to find the right time to take a connection which is driven multiple times.
        \end{itemize}
    \end{block} 
\end{frame}



\begin{frame}
    \frametitle{Keyboard Robot}
    \begin{block}{Problem}
        \begin{itemize}
            \item Given a 6x6 keyboard layout of letters and some text.
            \item Find a way to move 2 fingers simultaneously such that the time is minimal to type the given text.
        \end{itemize}
    \end{block} \pause
    \begin{block}{Insights}
        \begin{itemize}
            \item Insight \#1: the text is short, only 200 letters, so the maximum time is $(5+5)\cdot200 = 2000$
            \pause
            \item Insight \#2: we can simulate it, but need fast way of prioritising interesting states
            \pause
            \item Insight \#3: grid is unhelpful, save as basic graph instead: 
                {\tt dist[(from\_pos, to\_pos)] = distance}\\ 
                {\tt letter\_to\_pos[letter] = pos}\\
        \end{itemize}
    \end{block} 
\end{frame}
\begin{frame}
    \frametitle{Keyboard Robot}
    \begin{block}{Solution}
        \begin{itemize}
            \item Use a priority queue to track every "reasonable" state
            \pause
            \item Initial state is {\tt (0, 0, (0, 0), 0, (0, 0), 0)}
            \pause
            \item A state is {\tt(time, index, pos1, rem1, pos2, rem2)}
            \begin{enumerate}
                \item $time$ is the time since start of simulation
                \item $index$ is the index of the current letter to be typed in the text
                \item $pos_{i}$ is the position as a tuple of the $i$-th finger
                \item $rem_{i}$ is the remaining time to move for the $i$-th finger (if negative it means it has been idle for some time and can be moved retroactively)

            \end{enumerate}
            \pause
            \item From every state put 2 new states inside the priority queue: what if either finger 1 or finger 2 moves to the next letter
            \pause
            \item Only states where either $rem1$ or $rem2$ are 0 should be put in the priority queue
            \pause
            \item Simulate until some index is at the end of the char sequence
        \end{itemize}
    \end{block} 
\end{frame}

\begin{frame}
    \frametitle{Closing Remarks}
    \begin{itemize}
        \item Codeforces offers several online contests per week
        \pause
        \item There is a Github repository containing all problem code and our solutions
        \item Link: https://github.com/LiquidFun/FSR-Coding-Cup-2023
        \item Feel free to make a pull request with your own solutions so we can add them if you like.
        \pause
        \item We also opened a discord server: https://discord.gg/5xgkKyrf
    \end{itemize}
\end{frame}

\begin{frame}
    \frametitle{Closing Remarks}
    \begin{itemize}
        \item We hope you had fun today!
        \item Do you have any suggestions or thing we can improve on?
    \end{itemize}
\end{frame}

\begin{frame}
    \frametitle{Scoreboard Reveal}
\end{frame}

\begin{frame}
    \frametitle{Prizes}
    \begin{itemize}
        \item The first place receives a 150€ powerbank sponsored by Stackmeister GmbH.
        \item The second place receives a 15€ Amazon gift card per person
        \item The third place receives a 10€ Amazon gift card per person
        \item The fourth place receives a 5€ Amazon gift card per person
        \item The best Ersti team gets 10€ per person
        \item Also respect for everyone who chose to participate on a Saturday! :D
    \end{itemize}
\end{frame}

\end{document}
