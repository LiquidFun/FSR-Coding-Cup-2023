\documentclass{beamer}
\usetheme{Copenhagen}
%Information to be included in the title page:
\title{Sample title}
\author{Anonymous}
\institute{Overleaf}
\date{2021}

\begin{document}

\frame{\titlepage}

\begin{frame}
    \frametitle{Aquarium Maze}
    \begin{block}{Problem}
        \begin{itemize}
            \item Input: grid of $"."$ and $"\#"$ squares.
            \item Grid is filled with water from the top.
            \item Water can move down, left and right.
        \end{itemize}
    \end{block}
    \begin{block}{Solution}
        \begin{itemize}
            \item Can simulate water by starting at a top square and then traversing the graph e.g. using BFS or DFS
            \begin{enumerate}
                \item if square $== "."$ and not yet visited
                \item visit all neighbours recursively
                \item add 1 to awnser
            \end{enumerate}
        \end{itemize}
    \end{block}
    \begin{block}{Gotchas}
        \begin{itemize}
            \item Need to start once at every point on the top.
            \item Otherwise we might miss some air bubbles.
        \end{itemize}
    \end{block}
\end{frame}

\begin{frame}
    \frametitle{BicycleLock}
    \begin{block}{Problem}
        \begin{itemize}
            \item Input: Initial lock position $I$ and final lock position $F$ of length n.
            \item Move to final position by always turning two consecutive dials at once.
        \end{itemize}
    \end{block}
    \begin{block}{Solution}
        \begin{itemize}
            \item Dial 1 can only be turned by turning dials 1 and 2.
            \item It needs to be turned from $I_1$ to $I_2$.
            \item After turning that, we have a new subproblem of length $n-1$
            \item We can solve this recursively
        \end{itemize}
    \end{block}
    \begin{block}{Gotchas}
        \begin{itemize}
            \item Always two ways to turn dials: Clockwise or Anti-clockwise.
            \item Need to check if dial n is at the right positition in the end.
        \end{itemize}
    \end{block}
\end{frame}

\begin{frame}
    \frametitle{CompilersBrackets}
    \begin{block}{Problem}
        \begin{itemize}
            \item Check if given bracket pattern makes sense.
            \item $\{\{\}\{\}$ Does not make sense.
            \item $\{\{\}\{\{\}\}\}$ Does make sense.
        \end{itemize}
    \end{block}
    \begin{block}{Solution}
        \begin{itemize}
            \item Count number of currently open brackets.
            \item begin with open $= 0$
            \item $"\{" \rightarrow$ open$++$
            \item $"\}" \rightarrow$ open$--$
            \item pattern invalid if open $< 0$ at any time.
            \item pattern invalid if open $!= 0$ in the end.
        \end{itemize}
    \end{block}
\end{frame}

\begin{frame}
    \frametitle{DamConstruction}
    \begin{block}{Problem}
        \begin{itemize}
            \item Given $(n_1, n_2, n_4)$ lego bricks of size 1, 2 and 4.
            \item Build the highest wall of width $w$.
        \end{itemize}
    \end{block}
    \begin{block}{Solution}
        \begin{itemize}
            \item Should always use bricks of higher size first to maintain flexibility.
            \item Greedy solution by using $n_4$ bricks, then filling up with $n_2$, then with $n_1$.
        \end{itemize}
    \end{block}
\end{frame}

\begin{frame}
    \frametitle{ExtravagantVoyage}
    \begin{block}{Problem}
        \begin{itemize}
            \item Given $n$ items with happiness $H$ and volume $V$
            \item Choose items with cumulative weight $w$
            \item Also called 0-1-Knapsack
        \end{itemize}
    \end{block}
    \begin{block}{Solution}
        \begin{itemize}
            \item Recursive DP solution:
            \item Go through n items and start with remaining weight $r$ = $w$.
            \item Recursively solve:
            \begin{itemize}
            \item taking item: $r$ -= $W_i$; $h$ += $H_i$.
            \item leaving item: $r$, $h$ unchanged.
            \end{itemize}
            \item Save states in dp-table.
        \end{itemize}
    \end{block}
    \begin{block}{Gotchas}
        \begin{itemize}
            \item Not using a dp-table results in time limit exceeded.
        \end{itemize}
    \end{block}
\end{frame}

\begin{frame}
    \frametitle{FascinatingBooks}
    \begin{block}{Problem}
        \begin{itemize}
            \item Check if given list of strings contains every letter of the alphabet.
        \end{itemize}
    \end{block}
    \begin{block}{Solution}
        \begin{itemize}
            \item Can concatenate strings and solve for a single string.
            \item For each char:
            \item If char is letter: Add lowercase version to set.
            \item Check if length of set is 26.
        \end{itemize}
    \end{block}
    \begin{block}{Gotchas}
        \begin{itemize}
            \item Capitalization does not matter.
            \item Strings do not only contain letters
        \end{itemize}
    \end{block}
\end{frame}

\begin{frame}
    \frametitle{GoingHome}
    \begin{block}{Problem}
        \begin{itemize}
            \item 
        \end{itemize}
    \end{block}
    \begin{block}{Solution}
        \begin{itemize}
            \item 
        \end{itemize}
    \end{block}
    \begin{block}{Gotchas}
        \begin{itemize}
            \item 
        \end{itemize}
    \end{block}
\end{frame}

\begin{frame}
    \frametitle{HiddenWords}
    \begin{block}{Problem}
        \begin{itemize}
            \item TODO: make description more clear.
            \item Given n strings, find a string that contains all n strings in the right order and where no character is part of 3 strings.
        \end{itemize}
    \end{block}
    \begin{block}{Solution}
        \begin{itemize}
            \item Start with two first strings $X$ and $Y$.
            \item If $Y$ starts with $X$:
            \item Add $X$ to solutionword, continue with rest of $Y$ and next string.
            \item Else: Remove first letter of $X$ and repeat process.
        \end{itemize}
    \end{block}
\end{frame}

\begin{frame}
    \frametitle{IntuitiveCitations}
    \begin{block}{Problem}
        \begin{itemize}
            \item Given a list of names. Print the lexicographically smallest surname and add " et al.".
        \end{itemize}
    \end{block}
    \begin{block}{Solution}
        \begin{itemize}
            \item Remove part before space (prename).
            \item Find the lexicographically smallest string by sorting and taking the first element.
            \item print string + " et al.".
        \end{itemize}
    \end{block}
\end{frame}

\begin{frame}
    \frametitle{JollyFishing}
    \begin{block}{Problem}
        \begin{itemize}
            \item 
        \end{itemize}
    \end{block}
    \begin{block}{Solution}
        \begin{itemize}
            \item 
        \end{itemize}
    \end{block}
    \begin{block}{Gotchas}
        \begin{itemize}
            \item 
        \end{itemize}
    \end{block}
\end{frame}

\begin{frame}
    \frametitle{KeyboardRobot}
    \begin{block}{Problem}
        \begin{itemize}
            \item Given a 6x6 keyboard layout of letters and some text.
            \item Find a way to move 2 fingers simultaneously such that the time is minimal to type the given text.
        \end{itemize}
    \end{block}
    \begin{block}{Insights}
        \begin{itemize}
            \item Insight \#1: the text is short, only 200 letters, so the maximum time is $(5+5)\cdot200 = 2000$
            \item Insight \#2: we can simulate it, but need fast way of prioritising interesting states
        \end{itemize}
    \end{block}
    \begin{block}{Solution}
        \begin{itemize}
            \item Use a priority queue to track every "reasonable" reachable state
            \item A state is: (time, index in text, finger 1\&2 target position and remaining movement)
            \item From every state put 2 new states inside the priority queue: what if either finger 1 or 2 moves to the next letter
            \item Keep track of how much time the fingers are standing still, so that you can move them retroactively
        \end{itemize}
    \end{block}
\end{frame}

\begin{frame}
    \frametitle{LeaderboardPrediction}
    \begin{block}{Problem}
        \begin{itemize}
            \item Given the times you need to solve each of the n problems.
            \item Determine the minimal penalty you can get on the contest.
        \end{itemize}
    \end{block}
    \begin{block}{Solution}
        \begin{itemize}
            \item The time you needed for the first problem will be added to the penalty of all problems you solve.
            \item It is always best to solve shortest problems first.
            \item Greedy solution: Sort problems by length, then simulate.
        \end{itemize}
    \end{block}
\end{frame}

\end{document}
